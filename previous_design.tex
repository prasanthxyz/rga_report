\section{Report Grading App}
\subsection{Functionality}
Report Grading App an application for conducting lab examinations, assignments, cribs, and marks management without any hassle. There are four main operations which need to be handled in these:
\subsubsection{Setting questions}
Professors generally have their lab question papers in odt, doc, and pdf formats. The app currently allows uploading a question paper in odt format, which will be converted to HTML so that it can be shown directly in the website (Bodhitree). After the upload and conversion, professor is provided with an interface to edit the question paper online. There is a ``Student's View'' too, to check how it is rendered for students.\\
\\
\textbf{Template for Question Paper}\\
The question paper uploaded must be in ODT file format. A list of questions for which answering template is to be provided for the student is preceded
by a Question Tag as given below. The first list that appears after the Question Tag "Q" is
considered to be a question list. All other lists are considered as instruction lists.\\
....\\
....\\
....\\
Q\\
\textless Question Section Title\textgreater (No ordered or unordered lists here)\\
\begin{itemize}
	\item \textless Question1\textgreater
	\begin{itemize}
		\item \textless Question1.1\textgreater
		\item \textless Question1.2\textgreater
		\item ....
	\end{itemize}
	\item \textless Question2\textgreater
\end{itemize}
....

\subsubsection{Answering questions}
We have a student-interface to answer the questions online. A rich-text editor textbox will be associated with every question in the question paper where the students can write their answers. At deadline, the input textboxes will disappear showing only the questions and the answers as uneditable text. There is also a functionality to submit additional files if required.

\subsubsection{Evaluation}
The TA interface is similar to the student interface. It renders the question paper along with the responses provided by the student. Every question-answer combination is associated with an input field where the TA can enter the marks, and a rich-text editor textbox where comments can be added.

\subsubsection{Consolidation of the marks}
The instructor interface also provides a function to consolidate marks given to each student for an assignment into a csv file and  automatically upload it to a Google spreadsheet provided by the instructor. This makes it simple for the instructor to float the spreadsheet to the students for reviewing the marks awarded.


\subsection{Design}
\subsubsection{Backend Design}
The student-responses and instructor comments are stored in files. Each student/group has a predetermined path in the filesystem where their responses and the corresponding instructor comments are stored. These two files are simply read and inserted into the question paper and rendered as student view or TA view. This file-storage design is better than using a database for storing student-responses since we require multiple queries to get the required info. Here it is just a single file-read operation.

\subsubsection{Database Tables}
The following are the tables that were used in the application
\begin{itemize}
\item \textbf{Labs} - Table for storing the individual assignment details\\
The data stored in this table includes lab name, publish time, and soft and hard deadlines.

\item \textbf{SubmissionTime} - Table for storing the submission time of groups for the particular lab assignment\\
The data stored in this table includes a foreign key to the particular lab, a foreign key to the group which made the submission, the timestamp in which the paper was touched last, and the user-id of the group member who made the submission (if it was an explicit submission of exam).

\item \textbf{LabGroup} - Table for storing the details of group members of a group associated with a particular lab assignment\\
The table has a foreign key to the corresponding lab assignment, and the ID of the group.

\item \textbf{StudentGroup} - Table for mapping a student to a particular LabGroup\\
The table has foreign keys to User table and LabGroup table, mapping the user to a labgroup.
\end{itemize}
