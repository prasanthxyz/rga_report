\section{Facilitate Group Assignments}

\subsection{Problem Statement}
According to the current app design, exams are answered individually. Every student has a separate answer paper and marks. It is required a provision to facilitate group submission of an answer sheet, where multiple people can simultaneously write in the same answer sheet. The answer sheet marks should reflect on each group member's score card.

\subsection{Design and Implementation}
Two database tables are used to store student grouping information. The first table LabGroup stores various groups for a particular Lab. The second table StudentGroup maps students to LabGroups, thus defining a group of students that are a LabGroup -- writing exams as a group for a particular lab.\\

A helper function `student\_to\_group' was defined which takes a student-id, course-id and lab number, and returns the LabGroup ID. This was used to map students to their groups.

Previous design had one folder in the server per student (folder name is student id), which contains his/her answers as an xml file. In-order to facilitate group assignments, one folder should be assigned to a group (folder name should be group id to be unique). As the students answer the question paper, answers reach their group's folder. Simultaneous answering is possible.

\subsection{Problems Faced}
Multiple (till now the maximum observed was two) LabGroups were forming for the same StudentGroups when the server load was high. On stress-testing the group-creating URL, this problem was replicated.\\
This problem was resolved in the student-to-group helper function, where if multiple groups are found for a student in a particular lab, all the LabGroups (along with the StudentGroups) are deleted. We also set up a condition that unless a single Group is present for a student, the exam interface won't even be served. This ensures that no data (answers) is lost on the above said deletion.
