\section{Objectives}
\subsection{Warmup}
\begin{enumerate}
\item Understanding django-python framework
\item Understanding React framework
\item Understanding BodhiTree code structure and database schema
\end{enumerate}
\subsection{Goals}
\begin{enumerate}
\item Colour coding out-of-video quiz questions as `correctly answered', `attempted and wrong', and `not attempted'.
\item Colour coding the quiz modules as `fully attempted', `partially attempted' and `not attempted'.
\item A discussion box per video section, tagged by the video section.
\end{enumerate}

\section{Preliminary Work}
\subsubsection*{Understanding Django-Python Framework}
The online tutorials at \href{https://docs.djangoproject.com/en/1.8/intro/tutorial01}{djangoproject.com} and \href{http://www.tangowithdjango.com}{tangowithdjango.com} were used to learn django and create some sample web apps.
\subsubsection*{Understanding React Framework}
React.js is a javascript library for creating user interfaces, by Facebook. It uses composable components which are loaded individually, and will re-render automatically when their data changes. To understand and use React.js, I used the online tutorials from \href{https://facebook.github.io/react/docs/tutorial.html}{facebook.github.io}.
\subsubsection*{Understanding BodhiTree structure}
With the understanding I got from the above tutorials, I read through the django `database classes' used in BodhiTree and understood the basic DB tables used. Also, I read the `jsx' files which were the React.js code for rendering various applications (like a quiz, or a discussion forum).